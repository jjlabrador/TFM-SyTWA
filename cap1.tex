%%%%%%%%%%%%%%%%%%%%%%%%%%%%%%%%%%%%%%%%%%%%%%%%%%%%%%%%%%%%%%%%%%%%%%%%%%%%%
% Chapter 1: Introducción 
%%%%%%%%%%%%%%%%%%%%%%%%%%%%%%%%%%%%%%%%%%%%%%%%%%%%%%%%%%%%%%%%%%%%%%%%%%%%%%%

%---------------------------------------------------------------------------------
\section{Antecedentes}
\label{1:sec:1}

La World Wide Web está sujeta a un cambio continuo. La llegada de HTML5, la creciente importancia
de AJAX y de la programación en el lado del cliente, las nuevas fronteras de la Web Semántica, y la
explosión de las redes sociales son ejemplos de esta tendencia general.
\bigskip

Las aplicaciones web parecen evolucionar hacia entornos cada vez más ricos y flexibles en los que
los usuarios pueden acceder con facilidad a los documentos, publicar contenido, escuchar música, ver
vídeos, realizar dibujos e incluso jugar usando un navegador. Esta nueva clase de software ubicuo no
cesa de ganar momentum y promueve nuevas formas de interacción y cooperación.
\bigskip

Ante la rápida evolución del software, los sistemas de control de versiones han adquirido una mayor importancia dentro de la metodología del desarrollo del software: la gestión de las versiones del propio software se ha convertido en una actividad crítica. Estos sistemas han evolucionado a la par que el software, proporcionando nuevas funcionalidades y orientándose hacia la colaboración.

%---------------------------------------------------------------------------------
\section{Estado actual del arte}
\label{1:sec:2}

Actualmente, hay numerosos sistemas de control de versiones. Todos ellos proporcionan mecanismos de almacenamiento del código, de modificación y de consulta histórica del mismo, a la vez que proporcionan un entorno colaborativo en el que los usuarios pueden colaborar e interactuar entre sí.

En el caso particular de GitHub, además de proporcionar lo mencionado anteriormente, observando el creciente número de estudiantes que utiliza la plataforma, ha creado herramientas específicas para facilitar sus desarrollos (ej: Student Developer Pack) y provee a profesores de herramientas para gestionar dichos desarrollos (ej: GitHub Classrooms).

Sin embargo, estas herramientas de gestión de desarrollos requieren una administración interactiva por parte del profesor. No cuentan aún con funcionalidades de automatización de tareas.

%---------------------------------------------------------------------------------
\section{Objetivos y actividades a realizar}
\label{1:sec:3}

En este proyecto se persigue integrar los conocimientos adquiridos durante los estudios del Máster y,
en especial, del itinerario de Tecnologías de la Información para solucionar problemas actuales de aplicaciones y servicios Web.

Los objetivos propuestos para alcanzar en este Trabajo de Fin de Máster ha sido los siguientes:
\begin{itemize}
  \item Conocer, dominar y practicar con lenguajes y herramientas de desarrollo de aplicaciones Web
en el servidor.
  \item Conocer, dominar y practicar con diferentes lenguajes y librerías en el cliente.
  \item Conocer, practicar y dominar de herramientas de Desarrollo Dirigido por Pruebas en entornos
web.
  \item Conocer, practicar y dominar diferentes lenguajes de marcas y de estilo.
  \item Conocer, practicar y dominar diferentes mecanismos de despliegue.
  \item Conocer, practicar y familiarizarse con diferentes mecanismos de seguridad, autentificación
y autorización.
  \item Conocer, practicar y dominar diferentes herramientas colaborativas y de control de versiones.
  \item Conocer, practicar y dominar Metodologías Ágiles de desarrollo de software.
\end{itemize}
\bigskip

Y las actividades a realizar en el mismo son las que se describen a continuación:
\begin{itemize}
  \item Revisión bibliográfica y estado del arte.
  \item Desarrollar una herramienta de línea de comandos escrita en Node.js que permita automatizar tareas relacionadas con repositorios de GitHub:
  \begin{itemize}
    \item Autenticación con GitHub.
    \item Listar organizaciones, asignaciones y repositorios de GitHub del usuario.
    \item Automatizar la descarga de repositorios.
    \item Automatizar la ejecución de scripts en los repositorios (TDD, creación de entorno, evaluación de código...).
    \item Recopilar la información obtenida de la automatización de tareas y presentarla al usuario (PDF, HTML...).
  \end{itemize}
  \item Redacción de la memoria.
  \item Preparación de la presentación oral.
\end{itemize}

%---------------------------------------------------------------------------------
\section{Tecnología usada}
\label{1:sec:3}
 
Para llevar a cabo el desarrollo de esta herramienta se planteó realizar el desarrollo en \ceis{Node.js}, creando una librería modular que se pudiese instalar mediante el gestor de paquetes de Node.js (NPM).

\begin{figure}[H]
\begin{center}
\includegraphics[width=0.3\textwidth]{images/nodejs-logo}
\end{center}
\end{figure}

Además, se ha hecho uso de otras tecnologías enumeradas a continuación:
\begin{itemize}
  \item NPM            \includegraphics[width=0.15\textwidth]{images/npm}
  \item GitHub         \includegraphics[width=0.1\textwidth]{images/github}
  \item GitBook        \includegraphics[width=0.3\textwidth]{images/gitbook}
  \item Travis-CI      \includegraphics[width=0.3\textwidth]{images/travis-ci-logo}
\end{itemize}