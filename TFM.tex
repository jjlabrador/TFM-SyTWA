\documentclass[spanish,a4paper,14pt,oneside]{extreport}
\usepackage[utf8]{inputenc}
\usepackage[spanish]{babel}
%%%%%%%%%%%%%%%%%%%%%%%%%%%%%%%%%%%%%%%%%%%%%%%%%%%%%%%%%%%%%%%%%%%%%%%%%%%%%%%
% Next 3+3 lines select PDF or PS output respectively (comment as apropriate)
% To switch from PDF and PS comment/uncomment here
% and change correspondingly the Makefile
%
\usepackage[pdftex]{color}
\usepackage[pdftex]{graphicx}
\graphicspath{{FIGURES/}}
%\usepackage[dvips]{color}
%%\usepackage[dvips]{graphicx}
%\graphicspath{{FIGURES/}}
%%%%%%%%%%%%%%%%%%%%%%%%%%%%%%%%%%%%%%%%%%%%%%%%%%%%%%%%%%%%%%%%%%%%%%%%%%%%%%%
\usepackage{alltt}
\usepackage{algorithm}
\usepackage{algorithmic}
\usepackage{multirow}
\usepackage[top=2cm, bottom=2cm, left=2cm, right=2cm]{geometry}
\usepackage{hyperref}

% Símbolo del euro
\usepackage[official]{eurosym}

%%%%%%%%%%%%%%%%%%%%%%%%%%%%%%%%%%%%%%%%%%%%%%%%%%%%%%%%%%%%%%%%%%%%%%%%%%%%%%%

% Referenciar nombres de una lista de items
\makeatletter
\let\orgdescriptionlabel\descriptionlabel
\renewcommand*{\descriptionlabel}[1]{%
  \let\orglabel\label
  \let\label\@gobble
  \phantomsection
  \edef\@currentlabel{#1}%
  %\edef\@currentlabelname{#1}%
  \let\label\orglabel
  \orgdescriptionlabel{#1}%
}
\makeatother

% Indice de referencias cruzadas
\usepackage{makeidx}
\makeindex

%%%%%%%%%%%%%%%%%%%%%%%%%%%%%%%%%%%%%%%%%%%%%%%%%%%%%%%%%%%%%%%%%%%%%%%%%%
%listings
\usepackage{listings}
\usepackage{color}
\definecolor{lightgray}{rgb}{.9,.9,.9}
\definecolor{darkgray}{rgb}{.4,.4,.4}
\definecolor{purple}{rgb}{0.65, 0.12, 0.82}
\lstdefinelanguage{JavaScript}{
  keywords={do, if, in, for, let, new, try, var, case, else, enum, eval, null, this, true, void, with, await, break, catch, class, const, false, super, throw, while, yield, delete, export, import, public, return, static, switch, typeof, default, extends, finally, package, private, continue, debugger, function, arguments, interface, protected, implements, instanceof},
  morecomment=[l]{//},
  morecomment=[s]{/*}{*/},
  morestring=[b]',
  morestring=[b]",
  ndkeywords={class, export, boolean, throw, implements, import, this},
  keywordstyle=\color{blue}\bfseries,
  ndkeywordstyle=\color{darkgray}\bfseries,
  identifierstyle=\color{black},
  commentstyle=\color{purple}\ttfamily,
  stringstyle=\color{red}\ttfamily,
  sensitive=true
}

\lstset{
   language=JavaScript,
   backgroundcolor=\color{lightgray},
   extendedchars=true,
   basicstyle=\footnotesize\ttfamily,
   showstringspaces=false,
   showspaces=false,
   numbers=left,
   numberstyle=\footnotesize,
   numbersep=9pt,
   tabsize=2,
   breaklines=true,
   showtabs=false,
   captionpos=b
}

%%%%%%%%%%%%%%%%%%%%%%%%%%%%%%%%%%%%%%%%%%%%%%%%%%%%%%%%%%%%%%%%%%%%%%%%%%%%%%%

\newcommand{\SONY}{{\sc Sony}}
\newcommand{\INTEL}{\textsf{\textsc{I}ntel}}

%%% Traducimos el pseudocodigo
\renewcommand{\algorithmicwhile}{\textbf{mientras}}
\renewcommand{\algorithmicend}{\textbf{fin}}
\renewcommand{\algorithmicdo}{\textbf{hacer}}
\renewcommand{\algorithmicif}{\textbf{si}}
\renewcommand{\algorithmicthen}{\textbf{entonces}}
\renewcommand{\algorithmicrepeat}{\textbf{repetir}}
\renewcommand{\algorithmicuntil}{\textbf{hasta que}}
\renewcommand{\algorithmicelse}{\textbf{en otro caso}}
\renewcommand{\algorithmicfor}{\textbf{para}}

%\newcommand{\RETURN}{\textbf{retornar} }
\newcommand{\RET}{\STATE \textbf{retornar} }
\newcommand{\TO}{\textbf{hasta} }
\newcommand{\AND}{\textbf{y} }
\newcommand{\OR}{\textbf{o} }

%Comando para el índice de referencias cruzadas
\newcommand{\cei}[1]
  {\index{#1}\emph{#1}}
  
\newcommand{\ceis}[1]  
  {\index{#1}{\bfseries {#1}}}
  
\newcommand{\ceit}[1]  
  {\index{#1}{#1}}
  
%%%%%%%%%%%%%%%%% Creamos un entorno para listar código fuente %%%%%%%%%%%%%%%
\newenvironment{sourcecode}
{\begin{list}{}{\setlength{\leftmargin}{1em}}\item\scriptsize\bfseries}
{\end{list}}

\newenvironment{littlesourcecode}
{\begin{list}{}{\setlength{\leftmargin}{1em}}\item\tiny\bfseries}
{\end{list}}

\newenvironment{summary}
{\par\noindent\begin{center}\textbf{Abstract}\end{center}\begin{itshape}\par\noindent}
{\end{itshape}}

\newenvironment{keywords}
{\begin{list}{}{\setlength{\leftmargin}{1em}}\item[\hskip\labelsep \bfseries Keywords:]}
{\end{list}}

\newenvironment{palabrasClave}
{\begin{list}{}{\setlength{\leftmargin}{1em}}\item[\hskip\labelsep \bfseries Palabras clave:]}
{\end{list}}

%%%%%%%%%%%%%%%%%%%%%%%%%%%%%%%%%%%%%%%%%%%%%%%%%%%%%%%%%%%%%%%%%%%%%%%%%%%%%%%
\begin{document}
\renewcommand\listtablename{Índice de Tablas}      % Estos comandos (al haber cargado babel)
\renewcommand\listfigurename{Índice de Figuras}    % Han de ir inmediatamente después del begin{document}

%%%%%%%%%%%%%%%%%%%%%%%%%%%%%%%%%%%%%%%%%%%%%%%%%%%%%%%%%%%%%%%%%%%%%%%%%%%%%%%
% First Page
%%%%%%%%%%%%%%%%%%%%%%%%%%%%%%%%%%%%%%%%%%%%%%%%%%%%%%%%%%%%%%%%%%%%%%%%%%%%%%%

\pagestyle{empty}
\thispagestyle{empty}


\newcommand{\HRule}{\rule{\linewidth}{1mm}}
\setlength{\parindent}{0mm}
\setlength{\parskip}{0mm}

\vspace*{\stretch{0.5}}

%\centering{\includegraphics[width=0.3\textwidth]{images/logo_vertical}                 !!!
\begin{center}
	\includegraphics[width=0.3\textwidth]{images/logo_vertical}
\end{center}

\vspace*{\stretch{0.5}}
\begin{center}
{\Huge Trabajo de Fin de Máster}
{\Huge Sistemas y Tecnologías Web Aplicadas (SyTWA)}
\end{center}

\HRule
\begin{flushright}
        {\Huge Shell para corrección automática de repositorios de GitHub} \\[2.5mm]
        {\Large \textit{CLI tool for automatic correction of GitHub's repositories} .} \\[5mm]
        {\Large Juan José Labrador González} \\[5mm]


\end{flushright}
\HRule
\vspace*{\stretch{2}}
\begin{center}
  \Large La Laguna, \today
\end{center}

\setlength{\parindent}{5mm}

%%%%%%%%%%%%%%%%%%%%%%%%%%%%%%%%%%%%%%%%%%%%%%%%%%%%%%%%%%%%%%%%%%%%%%%%%%%%%%%
% Signature page (add the official stamp)
%%%%%%%%%%%%%%%%%%%%%%%%%%%%%%%%%%%%%%%%%%%%%%%%%%%%%%%%%%%%%%%%%%%%%%%%%%%%%%%
\newpage
%\cleardoublepage
\thispagestyle{empty}

D. {\bf Casiano Rodríguez León}, con DNI número 42.020.072-S
profesor
Titular de Universidad
adscrito al Departamento
de Ingeniería Informática y de Sistemas
de la Universidad de La Laguna, como tutor

\bigskip
\bigskip
{\bf C E R T I F I C A }

\bigskip
\bigskip
\bigskip
Que la presente memoria titulada:

\bigskip
``{\it Sistemas y Tecnologías Web Aplicadas. Shell para corrección automática de repositorios de GitHub.}''

\bigskip
\bigskip
\bigskip

\noindent ha sido realizada bajo su dirección por D. {\bf Juan José Labrador González},
con DNI número 78.729.778-L.

\bigskip
\bigskip

Y para que así conste, en cumplimiento de la legislación vigente y a los efectos
oportunos firman la presente en La Laguna a \today

%\cleardoublepage
\newpage
%%%%%%%%%%%%%%%%%%%%%%%%%%%%%%%%%%%%%%%%%%%%%%%%%%%%%%%%%%%%%%%%%%%%%%%%%%%%%%%
\thispagestyle{empty}

{ \flushright

\begin{LARGE}
Agradecimientos
\end{LARGE}

\hspace{3mm}

\begin{large}


\hspace{3mm}
La realización de esta asignatura de Trabajo de Fin de Máster no hubiera sido posible sin 
la ayuda de la Sección de Ingeniería Informática de la Escuela Superior de Ingeniería 
y Tecnología, que ha llevado a cabo todos los trámites necesarios.
\bigskip

\hspace{3mm}
Mención especial para mi familia, pareja y amigos, quienes me han alentado para no rendirme y lograr mis objetivos pese a las dificultades y contratiempos encontrados durante la realización de este Trabajo de Fin de Máster.
\bigskip

\hspace{3mm}
Y por último, especialmente agradecer a Casiano Rodríguez León su labor como tutor del 
Trabajo de Fin de Máster. Además de aprender muchísimo junto a él, me ha aconsejado, animado y resuelto mis 
dudas de manera incansable en la realización de este trabajo. Estoy seguro de que la experiencia y conocimientos adquiridos gracias a él, me ayudarán en mis próximos retos profesionales y personales.


\end{large}

}

%%%%%%%%%%%%%%%%%%%%%%%%%%%%%%%%%%%%%%%%%%%%%%%%%%%%%%%%%%%%%%%%%%%%%%%%%%%%%%%%%
\newpage

\begin{huge}
Licencia
\end{huge}

\bigskip
\bigskip
%\centering{\includegraphics[width=0.2\textwidth]{images/by-nc-sa_88x31}                      !!!
\begin{center}
	\includegraphics[width=0.2\textwidth]{images/by-nc-sa_88x31}
\end{center}

\begin{center}
{\Large \copyright~Esta obra está bajo una licencia de Creative Commons Reconocimiento-NoComercial-CompartirIgual 4.0 Internacional.
}
\end{center}


%%%%%%%%%%%%%%%%%%%%%%%%%%%%%%%%%%%%%%%%%%%%%%%%%%%%%%%%%%%%%%%%%%%%%%%%%%%%%%%
\newpage  %\cleardoublepage
\begin{abstract}
{\em

El objetivo de este Trabajo de Fin de Máster ha sido integrar los conocimientos adquiridos durante los estudios del Máster y,en especial, del itinerario de Tecnologías de la Información, aproximando al alumno a la resolución de 
problemas de aplicaciones Web y favoreciendo el desarrollo de destrezas propias de la Ingeniería Web: 
se centra en el aprendizaje y puesta en práctica de metodologías, aproximaciones, técnicas y herramientas 
para abordar la creciente complejidad de este tipo de aplicaciones en el marco de las metodologías ágiles. Cada vez ésta cobra más importancia, siendo constante el aumento del número de aplicaciones de escritorio, smartphones y tablets.
\bigskip

En este Trabajo de Fin de Máster se propone el desarrollo de un paquete Node.js (NPM) que facilite la descarga y corrección de repositorios GitHub de alumnos.
Existe un buen número de herramientas de Control de Versiones que permiten alojar proyectos software y agruparlos en organizaciones lógicas, pero carecen de mecanismos para automatizar funciones de uso cotidiano como la descarga de los mismos, la preparación del entorno de cada proyecto o la ejecución de pruebas.
\bigskip

En nuestra propuesta, se ha realizado una primera aproximación a la automatización de descargas y correcciones de repositorios, recopilando todos los datos inherentes de estas acciones y generando los informes correspondientes en formato PDF y HTML. Todo ello mediante un sencillo uso y sentando las bases para proporcionar más funcionalidades a la herramienta en un futuro próximo.
}

\begin{palabrasClave}
Consola, CLI, Shell, Terminal, Node.js, GitHub, Corrección, Automatización
\end{palabrasClave}

\end{abstract}
%%%%%%%%%%%%%%%%%%%%%%%%%%%%%%%%%%%%%%%%%%%%%%%%%%%%%%%%%%%%%%%%%%%%%%%%%%%%%%%

%%%%%%%%%%%%%%%%%%%%%%%%%%%%%%%%%%%%%%%%%%%%%%%%%%%%%%%%%%%%%%%%%%%%%%%%%%%%%%%
\newpage  %\cleardoublepage
\begin{summary}
{\em
Insert here an abstract in an EU language (preferably English)
}

\begin{keywords}
Console, CLI, Shell, Terminal, Node.js, GitHub, Correction, Automation
\end{keywords}

\end{summary}
%%%%%%%%%%%%%%%%%%%%%%%%%%%%%%%%%%%%%%%%%%%%%%%%%%%%%%%%%%%%%%%%%%%%%%%%%%%%%%%

%%%%%%%%%%%%%%%%%%%%%%%%%%%%%%%%%%%%%%%%%%%%%%%%%%%%%%%%%%%%%%%%%%%%%%%%%%%%%%%
\newpage{\pagestyle{empty}}
\thispagestyle{empty}

%%%%%%%%%%%%%%%%%%%%%%%%%%%%%%%%%%%%%%%%%%%%%%%%%%%%%%%%%%%%%%%%%%%%%%%%%%%%%%%


\pagestyle{myheadings} %my head defined by markboth or markright
% No funciona bien \markboth sin "twoside" en \documentclass, pero al
% ponerlo se dan un montón de errores de underfull \vbox, con lo que no se
% ha puesto.
\markboth{Juan José Labrador González}{Shell para corrección automática de repositorios de GitHub}

%%%%%%%%%%%%%%%%%%%%%%%%%%%%%%%%%%%%%%%%%%%%%%%%%%%%%%%%%%%%%%%%%%%%%%%%%%%%%%%
%Numeracion en romanos
\renewcommand{\thepage}{\roman{page}}
\setcounter{page}{1}

%%%%%%%%%%%%%%%%%%%%%%%%%%%%%%%%%%%%%%%%%%%%%%%%%%%%%%%%%%%%%%%%%%%%%%%%%%%%%%%

\tableofcontents

%%%%%%%%%%%%%%%%%%%%%%%%%%%%%%%%%%%%%%%%%%%%%%%%%%%%%%%%%%%%%%%%%%%%%%%%%%%%%%%
\newpage{\pagestyle{empty}}

\listoffigures

%%%%%%%%%%%%%%%%%%%%%%%%%%%%%%%%%%%%%%%%%%%%%%%%%%%%%%%%%%%%%%%%%%%%%%%%%%%%%%%
\newpage{\pagestyle{empty}}

\listoftables

%%%%%%%%%%%%%%%%%%%%%%%%%%%%%%%%%%%%%%%%%%%%%%%%%%%%%%%%%%%%%%%%%%%%%%%%%%%%%%%
\newpage{\pagestyle{empty}}

%%%%%%%%%%%%%%%%%%%%%%%%%%%%%%%%%%%%%%%%%%%%%%%%%%%%%%%%%%%%%%%%%%%%%%%%%%%%%%%
%Numeracion a partir del capitulo I
\renewcommand{\thepage}{\arabic{page}}
\setcounter{page}{1}


\chapter{Introducción}
\label{chapter:intro}

%%%%%%%%%%%%%%%%%%%%%%%%%%%%%%%%%%%%%%%%%%%%%%%%%%%%%%%%%%%%%%%%%%%%%%%%%%%%%
% Chapter 1: Introducción 
%%%%%%%%%%%%%%%%%%%%%%%%%%%%%%%%%%%%%%%%%%%%%%%%%%%%%%%%%%%%%%%%%%%%%%%%%%%%%%%

%---------------------------------------------------------------------------------
\section{Sección Uno}
\label{1:sec:1}

\begin{itemize}
  \item Item 1
  \item Item 2
  \item Item 3
  \item Item 4
\end{itemize}

%---------------------------------------------------------------------------------
\section{Sección Dos}
\label{1:sec:2}

\begin{itemize}
  \item Item 1
  \item Item 2
  \item Item 3
\end{itemize}

%---------------------------------------------------------------------------------
\section{Sección Tres}
\label{1:sec:3}

Bla, bla, bla

%---------------------------------------------------------------------------------
\section{Sección Cuatro}
\label{1:sec:4}

Bla, bla, bla

%------------------------------------------------------------------------------
%%%%%%%%%%%%%%%%%%%%%%%%%%%%%%%%%%%%% Figure %%%%%%%%%%%%%%%%%%%%%%%%%%%%%%%%%%
\begin{figure}[!htb]
   \centering
   \includegraphics[width=0.5\linewidth]{images/arbolbinario}
   \caption{Ejemplo de Figura}
   \label{}
\end{figure}
%------------------------------------------------------------------------------



%%%%%%%%%%%%%%%%%%%%%%%%%%%%%%%%%%%%%%%%%%%%%%%%%%%%%%%%%%%%%%%%%%%%%%%%%%%%%%%

\chapter{Desarrollo}
\label{chapter:dos}

%%%%%%%%%%%%%%%%%%%%%%%%%%%%%%%%%%%%%%%%%%%%%%%%%%%%%%%%%%%%%%%%%%%%%%%%%%%%%%%
% Chapter 2: Desarrollo
%%%%%%%%%%%%%%%%%%%%%%%%%%%%%%%%%%%%%%%%%%%%%%%%%%%%%%%%%%%%%%%%%%%%%%%%%%%%%%%

%++++++++++++++++++++++++++++++++++++++++++++++++++++++++++++++++++++++++++++++

En el capítulo anterior se ha descrito el estado del arte actual y se ha definido el Trabajo de Fin de Máster, especificado los objetivos, actividades a desarrollar y las tecnologías empleadas para su desarrollo. A continuación, se describirá la metodología de trabajo seguida.

%++++++++++++++++++++++++++++++++++++++++++++++++++++++++++++++++++++++++++++++

\section{Metodología usada}
\label{2:sec:1}

Se ha llevado a cabo una {\it metodología de trabajo ágil}, común en el campo de la Ingeniería Informática, con reuniones periódicas en las que se definían una serie de tareas u objetivos (iteración) y que se presentaban en la siguiente reunión. 
\bigskip

De este modo, con la entrega de prototipos funcionales de la aplicación, se han ido testeando, corrigiendo y mejorando las 
funcionalidades, al mismo tiempo que detectando problemas no contemplados en las fases previas de diseño.
\bigskip

Esta metodología, además, ha propiciado la generación de ideas que se han traducido en nuevas características.
\newpage

%---------------------------------------------------------------------------------
\subsection{GitHub}
\label{subsec:2.1.1}

Para llevar a cabo esta metodología, se ha usado GitHub como herramienta de Control de Versiones (CVS).
Todo el código implementado se alojaba en dicha herramienta, permitiendo así su cómoda modificación y actualización.

\begin{figure}[H]
\begin{center}
\includegraphics[width=0.9\textwidth]{images/github1}
\caption{Captura del repositorio del paquete NPM en GitHub}
\label{fig:github1}
\end{center}
\end{figure}
\newpage

El trabajo se dividía en ramas, de modo que la versión estable de la aplicación (rama \verb|master|) quedara aislada de la 
versión en desarrollo (rama \verb|develop|) y de la rama experimental (rama \verb|test|).


\begin{figure}[H]
\begin{center}
\includegraphics[width=0.47\textwidth]{images/github2}
\caption{Ramas del repositorio}
\label{fig:github2}
\end{center}
\end{figure}
\newpage

La documentación adicional para llevar a cabo los desarrollos de cada iteración, así como los problemas detectados, se anotaban en el apartado de \verb|issues| con el fin de que quedara constancia de ello y se reflejara el estado en el que se encontraba cada uno.

\begin{figure}[H]
\begin{center}
\includegraphics[width=1\textwidth]{images/github3}
\caption{Apartado de issues}
\label{fig:github3}
\end{center}
\end{figure}

%---------------------------------------------------------------------------------
\subsection{Travis-CI}
\label{subsec:2.1.2}

Como herramienta de integración continua, se ha utilizado Travis-CI, con el fin de asegurarnos el despliegue de la aplicación era satisfactorio tras cada cambio subido a la herramienta de control de versiones (GitHub).

\begin{figure}[H]
\begin{center}
\includegraphics[width=1.1\textwidth]{images/travis-ci}
\caption{Herramienta de integración continua}
\label{fig:travisci}
\end{center}
\end{figure}

%---------------------------------------------------------------------------------
\subsection{Experiencia de usuario}
\label{subsec:2.1.3}

Por otra parte, el tutor del Trabajo de Fin de Máster ha hecho pruebas reales con el resultado de cada iteración, actuando como {\it Product Owner}. 
\bigskip

De este modo, se comprobaba el funcionamiento de la aplicación en un entorno real y se recibía un valioso feedback para corregir problemas o hacer mejoras en las siguientes iteraciones.


%%%%%%%%%%%%%%%%%%%%%%%%%%%%%%%%%%%%%%%%%%%%%%%%%%%%%%%%%%%%%%%%%%%%%%%%%%%%%%%
\newpage{\pagestyle{empty}}
\thispagestyle{empty}

\chapter{Resultados}
\label{chapter:tres}

%%%%%%%%%%%%%%%%%%%%%%%%%%%%%%%%%%%%%%%%%%%%%%%%%%%%%%%%%%%%%%%%%%%%%%%%%%%%%%%
% Chapter 3: Resultados
%%%%%%%%%%%%%%%%%%%%%%%%%%%%%%%%%%%%%%%%%%%%%%%%%%%%%%%%%%%%%%%%%%%%%%%%%%%%%%%

%++++++++++++++++++++++++++++++++++++++++++++++++++++++++++++++++++++++++++++++

Finalizada la etapa de desarrollo del Trabajo de Fin de Máster, se procede a describir la herramienta implementada.


La herramienta se ha denominado ghhell, abreviatura de 'GitHub Shell'. Se ha publicado en NPM\cite{NPM} para su fácil distribución e instalación:

\begin{figure}[H]
\begin{center}
\includegraphics[width=0.8\textwidth]{images/npm1}
\caption{Página del gestor de paquetes NPM}
\label{fig:npm}
\end{center}
\end{figure}

Las funcionalidades implementadas en ghshell, se describen a continuación.

%---------------------------------------------------------------------------------
\section{Funcionalidades requeridas}
\label{3:sec:1}

%------------------------------------------------------------------------------------------------------------
\subsection{Autenticación con GitHub}
\label{subsec:3.1.1}
    
    Una vez que el usuario se autentifica con GitHub, se genera un token personal, que se usa posteriormente para acceder a la API de Github. Este token se almacena cifrado en el equipo del usuario, por lo que las siguientes ocasiones que utilice la herramienta no hará falta que vuelva a iniciar sesión:
        
        \begin{figure}[H]
		\begin{center}
		\includegraphics[width=0.7\textwidth]{images/ghshell1}
		\caption{Login de usuario}
		\label{fig:ghshell1}
		\end{center}
		\end{figure}
		
		\begin{figure}[H]
		\begin{center}
		\includegraphics[width=0.6\textwidth]{images/ghshell2-1}
		\caption{Usuario autenticado}
		\label{fig:ghshell2-1}
		\end{center}
		\end{figure}
		
		\begin{figure}[H]
		\begin{center}
		\includegraphics[width=1\textwidth]{images/ghshell2-3}
		\caption{Token personal en GitHub}
		\label{fig:ghshell2-3}
		\end{center}
		\end{figure}
		
		\begin{figure}[H]
		\begin{center}
		\includegraphics[width=0.6\textwidth]{images/ghshell2-4}
		\caption{Login automático una vez generado el token}
		\label{fig:ghshell2-4}
		\end{center}
		\end{figure}
		
	Si el usuario cierra sesión en la herramienta, se eliminará el token en GitHub y en el equipo:
	
		\begin{figure}[H]
		\begin{center}
		\includegraphics[width=0.6\textwidth]{images/ghshell2-2}
		\caption{Logout de usuario}
		\label{fig:ghshell2-2}
		\end{center}
		\end{figure}

%------------------------------------------------------------------------------------------------------------
\subsection{Listar organizaciones, asignaciones y repositorios de GitHub del usuario}
\label{subsec:3.1.2}   
    
    Con el comando 'orgs -l', se puede listar las organizaciones del usuario y usando 'repos -l', se listarán los repositorios del usuario. También se puede acceder 'virtualmente' a las organizaciones y listar los repositorios que contiene, así como las asignaciones.
\bigskip

    NOTA: se puede consultar toda la información referente a los comandos del programa en el Apéndice 2.
        
        \begin{figure}[H]
		\begin{center}
		\includegraphics[width=0.9\textwidth]{images/ghshell3}
		\caption{Lista de organizaciones del usuario}
		\label{fig:ghshell3}
		\end{center}
		\end{figure}
		
		\begin{figure}[H]
		\begin{center}
		\includegraphics[width=0.9\textwidth]{images/ghshell4}
		\caption{Lista de repositorios de una organización}
		\label{fig:ghshell4}
		\end{center}
		\end{figure}
		
		\begin{figure}[H]
		\begin{center}
		\includegraphics[width=0.9\textwidth]{images/ghshell5}
		\caption{Asignaciones dentro de otra organización}
		\label{fig:ghshell5}
		\end{center}
		\end{figure}
		
	También es posible acceder 'virtualmente' a los repositorios y realizar acciones sobre ellos:
	
		\begin{figure}[H]
		\begin{center}
		\includegraphics[width=1\textwidth]{images/ghshell5-1}
		\caption{Acceso a un repositorio de una organización}
		\label{fig:ghshell5-1}
		\end{center}
		\end{figure}

%------------------------------------------------------------------------------------------------------------
\subsection{Automatizar la descarga de repositorios}
\label{subsec:3.1.3}  
        	
    En función del contexto dónde nos encontremos dentro de la herramienta, podremos:
    \begin{itemize}
    	\item Clonar el repositorio en el que nos encontremos.
    	\item Clonar un repositorio determinado.
	    \item Clonar todos los repositorios que coincidan con una determinada expresión regular.
	    \item Clonar todos los repositorios de una asignación que coincidan con una determinada expresión
    \end{itemize}
    
    El clonado se realiza de manera asíncrona, por lo que podemos seguir trabajando mientras se clona(n) el/los repositorio(s). Se puede observar el estado de la clonación revisando el fichero de log que se genera: \textless nombre-repositorio \textgreater -clone.log.
    
    	\begin{figure}[H]
		\begin{center}
		\includegraphics[width=1\textwidth]{images/ghshell6-3}
		\caption{Clonado del repositorio donde nos encontramos}
		\label{fig:ghshell6-3}
		\end{center}
		\end{figure}	

    Si clonamos repositorios que pertenece a una organización, se creará una carpeta con el nombre de la organización y en su interior se guardarán los repositorios clonados.
    		
	Además, si clonamos repositorios que pertenecen a una asignación, también se creará una carpeta con el nombre de la asignación que los contendrá.
	
        \begin{figure}[H]
		\begin{center}
		\includegraphics[width=1\textwidth]{images/ghshell6-1}
		\caption{Clonado de asignaciones que coinciden con una expresión regular}
		\label{fig:ghshell6-1}
		\end{center}
		\end{figure}	
		
		\begin{figure}[H]
		\begin{center}
		\includegraphics[width=0.7\textwidth]{images/ghshell6-2}
		\caption{Resultado del clonado}
		\label{fig:ghshell6-2}
		\end{center}
		\end{figure}
	
%------------------------------------------------------------------------------------------------------------
\subsection{Automatizar la ejecución de scripts en los repositorios}
\label{subsec:3.1.3} 
	    
    En función del contexto donde nos encontremos dentro de la herramienta, podremos:
    \begin{itemize}
    	\item Ejecutar un script en el repositorio en el que nos encontremos.
    	\item Ejecutar un script en un determinado repositorio.
	    \item Ejecutar un script en todos los repositorios que coincidan con una determinada expresión regular.
	    \item Ejecutar un script en todos los repositorios de una asignación coincidan con una determinada expresión regular.
    \end{itemize}
    
	La ruta del fichero del script puede ser absoluta o relativa. Estos scripts puede ser de cualquier tipo: TDD, creación de entorno, evaluación de código...
\bigskip

	La ejecución de cada script se ejecuta en un proceso hijo independiente pero, a diferencia del clonado, el script se ejecuta línea a línea de manera síncrona. Se puede observar el estado de la ejecución del script y los resultados revisando el fichero de log que se genera: \textless nombre-repositorio \textgreater - \textless nombre-script \textgreater .log
    	
    	\begin{figure}[H]
		\begin{center}
		\includegraphics[width=1\textwidth]{images/ghshell7-3}
		\caption{Ejecución del script 'install.sh' en el repositorio actual}
		\label{fig:ghshell7-3}
		\end{center}
		\end{figure}
		
        \begin{figure}[H]
		\begin{center}
		\includegraphics[width=1\textwidth]{images/ghshell7-1}
		\caption{Ejecución del script 'install.sh' en asignaciones que coinciden con una expresión regular}
		\label{fig:ghshell7-1}
		\end{center}
		\end{figure}	
		
		\begin{figure}[H]
		\begin{center}
		\includegraphics[width=0.7\textwidth]{images/ghshell7-2}
		\caption{Resultado de la ejecución del script 'install.sh'}
		\label{fig:ghshell7-2}
		\end{center}
		\end{figure}
		
%------------------------------------------------------------------------------------------------------------
\subsection{Recopilar la información obtenida de la automatización de tareas}
\label{subsec:3.1.4}

    Una vez ejecutados los scripts necesarios para evaluar un determinado repositorio, es posible generar un GitBook con el resultado de la ejecución de los mismos. Este libro se genera en formato PDF y en HTML.
\bigskip
    
    En función del contexto dónde nos encontremos dentro de la herramienta, podremos:
    \begin{itemize}
    	\item Crear un GitBook en el repositorio en el que nos encontremos.
    	\item Crear un GitBook en un determinado repositorio.
	    \item Crear un GitBook en todos los repositorios que coincidan con una determinada expresión regular.
	    \item Crear un GitBook en todos los repositorios de una asignación coincidan con una determinada expresión regular.
    \end{itemize}
        
        \begin{figure}[H]
		\begin{center}
		\includegraphics[width=1\textwidth]{images/ghshell8-3}
		\caption{Creación del Gitbook en el repositorio actual}
		\label{fig:ghshell8-3}
		\end{center}
		\end{figure}
		
        \begin{figure}[H]
		\begin{center}
		\includegraphics[width=1\textwidth]{images/ghshell8-1}
		\caption{Creación del Gitbook en asignaciones que coinciden con una expresión regular}
		\label{fig:ghshell8-1}
		\end{center}
		\end{figure}	
		
		\begin{figure}[H]
		\begin{center}
		\includegraphics[width=0.7\textwidth]{images/ghshell8-2}
		\caption{Resultado de la creación del Gitbook}
		\label{fig:ghshell8-2}
		\end{center}
		\end{figure}
		
        		        		[ Imagen HTML]
        		        		        		[ Imagen PDF]
\newpage
%---------------------------------------------------------------------------------
\section{Funcionalidades extra}
\label{3:sec:2}

Además de las funcionales solicitadas en este Trabajo de Fin de Máster, se han añadido una serie de funcionalidades extra que, a pesar de no ser requeridas, brindan al usuario de una mejor experiencia de uso del programa:

\begin{itemize}
	\item Autocompletado de los comandos disponibles en función del contexto donde nos encontremos (nivel principal, organización o repositorio.
	\item Opción de ayuda que muestra la descripción de los comandos y cómo se utilizan. Esta ayuda varía dependiendo del contexto donde nos encontremos.
	\item Opción de visualizar el directorio de trabajo donde se ha ejecutado el programa. Útil para determinar rutas relativas de los scripts que se desean ejecutar.
	\item Opción para conocer el propietario de cada repositorio. En el caso de que el repositorio pertenezca a una organización, mostrará los contribuyentes de ese repositorio.
\end{itemize}

NOTA: se puede consultar toda la información referente a los comandos del programa en el Apéndice 2.

%---------------------------------------------------------------------------------
\section{Problemas encontrados y soluciones}
\label{3:sec:3}

A continuación se detallan los problemas encontrados durante la implementación de la herramienta y las soluciones encontradas para los mismos:

%---------------------------------------------------------------------------------
\subsection{Asincronía}
\label{subsec:3.3.1}

Una de las características más importantes del lenguaje Node.js es la asincronía. Usa un modelo de operaciones de entrada/salida sin bloqueo y orientado a eventos, que lo hace ligero y eficiente. Sin embargo, algunas acciones que debía realizar esta herramienta debían de ser síncronas. Ej: login del usuario y ejecución de scripts.
\bigskip

{\normalsize {\bfseries Solución}}
\bigskip

La solución a este comportamiento pasó por realizar un amplio estudio de la documentación para usar mecanismos que permitieran bloquear la ejecución de la herramienta en las partes que deseábamos. Los mecanismos usados han sido:

\begin{itemize}
	\item Funciones síncronas del propio lenguaje.
	\item Promesas
	\item Métodos async/await
	\item Librerías con métodos implementados de manera síncrona.
\end{itemize}

%---------------------------------------------------------------------------------
\subsection{Autocompletado de comandos}
\label{subsec:3.3.2}

Para el manejo de los flujos de lectura y escritura de la herramienta, se ha utilizado la interfaz nativa de Node.js (Readline). Esta interfaz provee de una función de autocompletado para el texto que escribe el usuario.

Sin embargo, sólo funciona con la primera palabra (comando) que escribe. Tras investigar al respecto y buscar posibles librerías alternativas, no existía ninguna solución que corrigiera este comportamiento.
\bigskip

{\normalsize {\bfseries Solución}}
\bigskip

Realizando numerosas pruebas, se halló una manera propia de conseguir completar más de un comando en la misma línea. Cuando realice los test de aceptación pertinentes requeridos por la comunidad de Node, solicitaré un Pull Request a su repositorio con esta mejora.


%---------------------------------------------------------------------------------
\section{Perfil del usuario de ghshell}
\label{3:sec:4}

El uso de ghshell está especialmente dirigido a un determinado grupo de profesores: nos referimos al perfil de un profesor, principalmente docente en alguna rama de Ingeniería, con conocimientos avanzados en programación y en herramientas de control de versiones.

No obstante, ya que la curva de aprendizaje de ghshell no es excesiva y dado que el uso de las herramientas de control de versiones no se limita exclusivamente a repositorios de código fuente, se puede extender su uso para el resto de profesorado y usuarios con otros roles. Basta con tener claras unas nociones básicas de informática, junto con la lectura y asimilación previa de la documentación de la herramienta.


%%%%%%%%%%%%%%%%%%%%%%%%%%%%%%%%%%%%%%%%%%%%%%%%%%%%%%%%%%%%%%%%%%%%%%%%%%%%%%%
\newpage{\pagestyle{empty}}
\thispagestyle{empty}

\chapter{Conclusiones y líneas futuras}
\label{chapter:Conclusiones}

%%%%%%%%%%%%%%%%%%%%%%%%%%%%%%%%%%%%%%%%%%%%%%%%%%%%%%%%%%%%%%%%%%%%%%%%%%%%%
% Chapter 4: Conclusiones y Trabajos Futuros 
%%%%%%%%%%%%%%%%%%%%%%%%%%%%%%%%%%%%%%%%%%%%%%%%%%%%%%%%%%%%%%%%%%%%%%%%%%%%%%%

%++++++++++++++++++++++++++++++++++++++++++++++++++++++++++++++++++++++++++++++

Desde hace unos años hasta ahora, ha tenido lugar un enorme crecimiento de las herramientas de control de versiones. Se han convertido en una herramienta imprescindible en la metodologías de desarrollo del software y las instituciones de enseñanza saben que incorporarlas a sus sistemas educativos es clave para ofrecer un servicio puntero y de calidad.
\bigskip

Ésto es lo que se pretende con la herramienta obtenida tras la realización de este Trabajo de Fin de Máster: que sea posible su implantación dentro del marco académico de la Universidad de La Laguna, partiendo de la premisa de que, actualmente, el desarrollo de un proyecto software sin tener detrás un sistema de control de versiones, no es viable.
\bigskip

La automatización de las tareas de clonado y ejecución de pruebas facilitaría al profesor, en primera instancia, la corrección de las prácticas y proyectos de los alumnos. El ahorro de tiempo de ejecutar estas tareas manualmente es considerable, teniendo en cuenta el número de prácticas que realiza cada alumno por asignatura. Esta enorme carga de trabajo del profesor puede ser aprovechada en otros ámbitos docentes.
\bigskip

Por otra parte, esta herramienta sienta las bases a posibles desarrollos futuros, ampliando las funcionalidades de la misma. Se ha desarrollado pensando en su posible escalabilidad y ya que cuenta con toda la estructura base creada (autentificación de usuarios, clonado, ejecución y reporte de resultados), se pueden añadir funcionalidades sin demasiado esfuerzo.
\newpage

Para concluir, podemos afirmar que los objetivos marcados al comienzo de este Trabajo de Fin de Máster han sido cumplidos y las principales líneas de desarrollo a continuar podrían ser las enumeradas a
continuación:  

\begin{itemize}
	\item Dotar de más funcionalidad de GitHub a la herramienta:
	\begin{itemize}
		\item Subir cambios a los repositorios (git push).
		\item Crear issues.
		\item Gestionar Pull Requests.
		\item Buscar repositorios.
		\item Gestión de permisos de usuarios a repositorios y organizaciones.
		\item Gestionar Classrooms.
	\end{itemize}
	\item Enriquecer el formato de la documentación generada.
	\item Realizar despliegues locales de aplicaciones web (como procesos hijos de la herramienta).
\end{itemize}



%%%%%%%%%%%%%%%%%%%%%%%%%%%%%%%%%%%%%%%%%%%%%%%%%%%%%%%%%%%%%%%%%%%%%%%%%%%%%%%
\newpage{\pagestyle{empty}}
\thispagestyle{empty}

\chapter{Summary and Conclusions }
\label{chapter:ingles}

%%%%%%%%%%%%%%%%%%%%%%%%%%%%%%%%%%%%%%%%%%%%%%%%%%%%%%%%%%%%%%%%%%%%%%%%%%%%%
% Chapter 5: Summary and Conlusions
%%%%%%%%%%%%%%%%%%%%%%%%%%%%%%%%%%%%%%%%%%%%%%%%%%%%%%%%%%%%%%%%%%%%%%%%%%%%%%%

%++++++++++++++++++++++++++++++++++++++++++++++++++++++++++++++++++++++++++++++

From a few years to now, it has had a enourmous growth of Version Control tools. It has become in an indispensable tool in the software development methodologies and all the teaching institutions know that incorporing them to their education systems is fundamental to providing a quality service.
\bigskip

This is what is intended with the developed tool after the completion of this Master's Degree Final Project: that it would be deployed at University of La Laguna basing on the premise that, currently, the development of a software project without using a version control system, is not viable. 
\bigskip

The automation of cloning and execution tasks would facilitate, in first instance, the students' projects correction. The amount of saved time per the teacher is considarable, keeping in mind the large amount of practices of each student per subject. This huge teacher's workload could be use in another academic scopes.
\bigskip

By the other hand, this tool set the bases to future improvements, like enlarge their own functionalities. It has been developed thinking about the scalability and, as it has all the structure created (user's authentication, cloning, execution and reporting), it's possible to add improvements without too effort.
\newpage

In conclusion, we can affirm that all the goals established at the beggining of the subject have been fulfilled and the next development guidelines could be:


\begin{itemize}
	\item Create scripts in \verb|Bash| to evaluate applications (installation of dependencies, code quality check and running tests) in several programming languages: \verb|Node.js|, \verb|C++|, \verb|Ruby|, \verb|Python|, etc.
	\bigskip
	
	This collections of scripts may be incorporated in the tool distribution or be supplied as standalone's applications which facilitate the use of \verb|ghshell|.
	
	\item Provide support to execute scripts written in anothers programming languages: Ruby, Python...
	\item Create {\it issues} in repositories with the results of scripts' executions.
	\item Create branches in repositories with the results of scripts' executions.
\end{itemize}


%%%%%%%%%%%%%%%%%%%%%%%%%%%%%%%%%%%%%%%%%%%%%%%%%%%%%%%%%%%%%%%%%%%%%%%%%%%%%%%
\newpage{\pagestyle{empty}}
\thispagestyle{empty}

\chapter{Presupuesto}
\label{chapter:presupuesto}

%%%%%%%%%%%%%%%%%%%%%%%%%%%%%%%%%%%%%%%%%%%%%%%%%%%%%%%%%%%%%%%%%%%%%%%%%%%%%
% Chapter 6: Presupuesto
%%%%%%%%%%%%%%%%%%%%%%%%%%%%%%%%%%%%%%%%%%%%%%%%%%%%%%%%%%%%%%%%%%%%%%%%%%%%%%%

%++++++++++++++++++++++++++++++++++++++++++++++++++++++++++++++++++++++++++++++

En este capítulo se especifica un presupuesto que indica cuánto costaría realizar este Trabajo de Fin de Máster
si se tratase de un trabajo encargado por un cliente.

%---------------------------------------------------------------------------------
\section{Introducción y coste por hora}
\label{6:sec:1}

Se definirá una tabla con la lista de actividades realizadas en este Trabajo de Fin de Máster. Otra columna indicará la duración en horas que se han empleado para dicha actividad junto con el precio por hora calculado.
\bigskip

El precio por hora que se considerará en este presupuesto es de 30\euro{}/hora.
\newpage
    
%---------------------------------------------------------------------------------
\section{Funcionalidades requeridas}
\label{6:sec:2}

%--------------------------------------------------------------------------
\begin{table}[!ht]
\begin{center}
\begin{tabular}{|p{80mm}|p{25mm}|p{20mm}|} \hline 
\textbf{Actividad} & \textbf{Duración} & \textbf{Precio} \\ \hline

Autenticación con GitHub &
xx horas &
xx \euro{}
\\
\hline

Listar organizaciones, asignaciones y repositorios &
xx horas &
xx \euro{}
\\
\hline

Automatizar la descarga de repositorios &
xx horas &
xx \euro{}
\\
\hline

Automatizar la ejecución de scripts en los repositorios &
xx horas &
xx \euro{}
\\
\hline

Exportar la información obtenida de la automatización de tareas &
xx horas &
xx \euro{}
\\
\hline \hline

{\bfseries Subtotal} &
{\bfseries xx horas} &
{\bfseries xx \euro{}}
\\
\hline

\end{tabular}
\end{center}
\caption{Tabla de actividades, duración y precios de las funcionalidades requeridas}
\label{table:resOthers1}
\end{table}



%---------------------------------------------------------------------------------
\section{Funcionalidades extra}
\label{6:sec:3}

%--------------------------------------------------------------------------
\begin{table}[!ht]
\begin{center}
\begin{tabular}{|p{80mm}|p{25mm}|p{20mm}|} \hline 
\textbf{Actividad} & \textbf{Duración} & \textbf{Precio} \\ \hline

Autocompletado de comandos según contexto &
xx horas &
xx \euro{}
\\
\hline

Opción de ayuda según contexto &
xx horas &
xx \euro{}
\\
\hline

Visualización del directorio de trabajo actual &
xx horas &
xx \euro{}
\\
\hline

Opción para conocer propietarios del repositorio &
xx horas &
xx \euro{}
\\
\hline \hline

{\bfseries Subtotal} &
{\bfseries xx horas} &
{\bfseries xx \euro{}}
\\
\hline

\end{tabular}
\end{center}
\caption{Tabla de actividades, duración y precios de las funcionalidades extra}
\label{table:resOthers2}
\end{table}


\newpage
%---------------------------------------------------------------------------------
\section{Coste y duración total}
\label{6:sec:4}

%--------------------------------------------------------------------------
\begin{table}[!ht]
\begin{center}
\begin{tabular}{|p{80mm}|p{25mm}|p{20mm}|} \hline 
\textbf{Actividad} & \textbf{Duración} & \textbf{Precio} \\ \hline

Funcionalidades requeridas &
xx horas &
xx \euro{}
\\
\hline

Funcionalidades extra &
xx horas &
xx \euro{}
\\
\hline \hline

{\bfseries Total} &
{\bfseries xx horas} &
{\bfseries xx \euro{}}
\\
\hline

\end{tabular}
\end{center}
\caption{Precio y duración total}
\label{table:resOthers3}
\end{table}




%%%%%%%%%%%%%%%%%%%%%%%%%%%%%%%%%%%%%%%%%%%%%%%%%%%%%%%%%%%%%%%%%%%%%%%%%%%%%%%
\newpage{\pagestyle{empty}}
\thispagestyle{empty}
\begin{appendix}

\chapter{Glosario}
\label{appendix:1}
{\bfseries {\Huge A}}\label{Apendice1:A}
\bigskip
\bigskip

\begin{description}
  \item[\underline{AJAX}\label{apend1:ajax}]: acr\'onimo de \textit{Asynchronous JavaScript And XML} (JavaScript as\'{\i}ncrono y XML). Es una t\'ecnica de desarrollo web para crear aplicaciones 
  interactivas o RIA (\textit{Rich Internet Applications}). Estas aplicaciones se ejecutan en el cliente, es decir, en el navegador de los usuarios mientras se 
  mantiene la comunicaci\'on as\'{\i}ncrona con el servidor en segundo plano. De esta forma es posible realizar cambios sobre las p\'aginas sin necesidad de 
  recargarlas, mejorando la interactividad, velocidad y usabilidad en las aplicaciones.
  \bigskip
\end{description}

\begin{description}
  \item[\underline{API}\label{apend1:api}]: (\textit{Application Programming Interface} o Interfaz de Programaci\'on de Aplicaciones). Conjunto de funciones y procedimientos o m\'etodos que 
  ofrece cierta librer\'{\i}a para ser utilizados por otro software como una capa de abstracci\'on. 
  \bigskip
\end{description}

\begin{description}
  \item[\underline{Asíncrono}\label{apend1:asincrono}]: 
  \bigskip
\end{description}

\begin{description}
  \item[\underline{Asignación}\label{apend1:asignacion}]: 
  \bigskip
\end{description}

\begin{description}
  \item[\underline{Async/Await}\label{apend1:async-await}]: 
  \bigskip
\end{description}


\bigskip
{\bfseries {\Huge C}}\label{Apendice1:C}
\bigskip
\bigskip

\begin{description}
   \item[\underline{CVS}\label{apend1:cvs}]: (\textit{Control Versioning System} o Sistema de Control de Versiones). Aplicaci\'on inform\'atica que implementa un sistema de control de 
  versiones: mantiene el registro de todo el trabajo y los cambios en los ficheros (c\'odigo fuente principalmente) que forman un proyecto y permite que distintos desarrolladores 
  (potencialmente situados a gran distancia) colaboren.
  \bigskip
\end{description}


{\bfseries {\Huge G}}\label{Apendice1:G}
\bigskip
\bigskip

\begin{description}
  \item[\underline{GitBook}\label{apend1:gitbook}]: 
  \bigskip
\end{description}

\begin{description}
  \item[\underline{GitHub}\label{apend1:github}]: forja para alojar proyectos utilizando el Sistema de Control de Versiones {\bfseries Git}. Para m\'as informaci\'on, visitar {\small 
  \url{https://github.com}}.
  \bigskip
\end{description}

\begin{description}
  \item[\underline{GitHub Classroom}\label{apend1:github-classroom}]:
  \bigskip
\end{description}

\bigskip
{\bfseries {\Huge H}}\label{Apendice1:H}
\bigskip
\bigskip

\begin{description}
  \item[\underline{HTML5}\label{apend1:html}]: (\textit{HyperText Markup Language}). Lenguaje de marcado para la elaboraci\'on de p\'aginas web. Es un est\'andar que sirve de referencia para la 
  elaboraci\'on de p\'aginas web definiendo una estructura b\'asica y un c\'odigo para la definici\'on del contenido de la misma.
  \bigskip
\end{description}

\bigskip
\newpage

{\bfseries {\Huge J}}\label{Apendice1:J}
\bigskip
\bigskip

\begin{description}
  \item[\underline{JavaScript}\label{apend1:js}]: lenguaje de programaci\'on interpretado. Se define como orientado a objetos, basado en prototipos, imperativo, d\'ebilmente tipado y 
  din\'amico. Se utiliza principalmente en su forma del lado del cliente (\textit{client-side}), implementado como parte de un navegador web permitiendo mejoras en la interfaz de usuario y p\'aginas 
web din\'amicas.
  \bigskip
\end{description}

\bigskip
{\bfseries {\Huge M}}\label{Apendice1:M}
\bigskip
\bigskip

\begin{description}
  \item[\underline{Metodologias \'agiles}\label{apend1:ma}]: conjunto de m\'etodos de ingenier\'{\i}a del software basados en el desarrollo iterativo e incremental, donde los requisitos y 
  soluciones evolucionan mediante la colaboraci\'on de grupos auto organizados y multidisciplinarios. Se caracterizan adem\'as por la minimizaci\'on de riesgos desarrollando software en
  iteraciones cortas de tiempo.
  \bigskip
\end{description}

\bigskip
{\bfseries {\Huge N}}\label{Apendice1:N}
\bigskip
\bigskip

\begin{description}
  \item[\underline{Node.js}\label{apend1:node}]:
  \bigskip
\end{description}

\begin{description}
  \item[\underline{NPM}\label{apend1:npm}]:
  \bigskip
\end{description}

\bigskip
{\bfseries {\Huge O}}\label{Apendice1:O}
\bigskip
\bigskip

\begin{description}
  \item[\underline{Organización}\label{apend1:organizacion}]:
  \bigskip
\end{description}

\bigskip
{\bfseries {\Huge P}}\label{Apendice1:P}
\bigskip
\bigskip

\begin{description}
  \item[\underline{Promesa}\label{apend1:promesa}]:
  \bigskip
\end{description}

\bigskip
{\bfseries {\Huge R}}\label{Apendice1:R}
\bigskip
\bigskip

\begin{description}
  \item[\underline{Repositorio}\label{apend1:repositorio}]:
  \bigskip
\end{description}

{\bfseries {\Huge S}}\label{Apendice1:S}
\bigskip
\bigskip

\begin{description}
  \item[\underline{Student Developer Pack}\label{apend1:sdp}]: 
  \bigskip
\end{description}

\begin{description}
  \item[\underline{Síncrono}\label{apend1:sincrono}]: 
  \bigskip
\end{description}

{\bfseries {\Huge T}}\label{Apendice1:T}
\bigskip
\bigskip

\begin{description}
  \item[\underline{Travis-CI}\label{apend1:travis}]:
  \bigskip
\end{description}

\begin{description}
  \item[\underline{TDD}\label{apend1:tdd}]: (\textit{Test-Driven Development} o Desarrollo Dirigido por Pruebas). Pr\'actica de programaci\'on que involucra otras dos pr\'acticas: escribir las 
  pruebas primero (\textit{Test First Development}) y Refactorizaci\'on de c\'odigo (\textit{Refactoring}).
  \bigskip
\end{description}

\begin{description}
  \item[\underline{Token}\label{apend1:token}]:
  \bigskip
\end{description}

\bigskip
{\bfseries {\Huge W}}\label{Apendice1:W}
\bigskip
\bigskip

\begin{description}
  \item[\underline{Web sem\'antica}\label{apend1:web}]: idea de a\~{n}adir metadatos sem\'anticos y ontol\'ogicos a la World Wide Web. Esas informaciones adicionales, que describen el contenido, 
  el significado y la relaci\'on de los datos, se deben proporcionar de manera formal, para que sea posible evaluarlas autom\'aticamente por m\'aquinas de procesamiento. El objetivo es mejorar 
  Internet ampliando la interoperabilidad entre los sistemas inform\'aticos usando {\bfseries agentes inteligentes}, es decir, programas en las computadoras que buscan informaci\'on sin necesidad de 
  interacci\'on humana.
  \bigskip
\end{description}

\begin{description}
  \item[\underline{World Wide Web}\label{apend1:www}]: (WWW). Sistema de distribuci\'on de documentos de hipertexto o hipermedios interconectados y accesibles v\'{\i}a Internet. Con un navegador 
  web, un usuario visualiza sitios web compuestos de p\'aginas web que pueden contener texto, im\'agenes, v\'{\i}deos u otros contenidos multimedia, y navega a trav\'es de esas p\'aginas usando 
hiperenlaces.
  \bigskip
\end{description}

\chapter{Guía de uso}
\label{appendix:2}
El objetivo de esta guía de usuario es proporcionar a los usuarios un ejemplo para la puesta a punto y ejecución de las 
funcionalidades implementadas en el paquete NPM ghshell durante el Trabajo de Fin de Máster.

%---------------------------------------------------------------------------------
\section{Instalación}
\label{Apendice2:instalacion}

\subsection{Requisitos}
\label{subsec:b.1.1}

Node.js \textgreater = 8

\subsection{Dependencias}
\label{subsec:b.1.2}

Gitbook
Calibre

\subsection{Instalación}
\label{subsec:b.1.3}

Para instalar el paquete, basta con ejecutar el siguiente comando:
\begin{verbatim}
[~]$ npm install ghshell -g
\end{verbatim}

%---------------------------------------------------------------------------------
%---------------------------------------------------------------------------------

\section{Ejecución}
\label{Apendice2:ejecucion}

\begin{lstlisting}[language=JavaScript]
var foo = function(){
console.log('foo');
}
foo();
\end{lstlisting}
\bigskip

%---------------------------------------------------------------------------------
\subsection{Otras consideraciones}
\label{subsec:Apendice2.1}

Para que

\end{appendix}

%%%%%%%%%%%%%%%%%%%%%%%%%%%%%%%%%%%%%%%%%%%%%%%%%%%%%%%%%%%%%%%%%%%%%%%%%%%%%%%
\clearpage
\phantomsection

\addcontentsline{toc}{chapter}{Índice alfabético}
\printindex

%%%%%%%%%%%%%%%%%%%%%%%%%%%%%%%%%%%%%%%%%%%%%%%%%%%%%%%%%%%%%%%%%%%%%%%%%%%%%%%
\addcontentsline{toc}{chapter}{Bibliografía}
\bibliographystyle{plain}%{ieeetr}

\bibliography{TFM}
\nocite{*}

%%%%%%%%%%%%%%%%%%%%%%%%%%%%%%%%%%%%%%%%%%%%%%%%%%%%%%%%%%%%%%%%%%%%%%%%%%%%%%%

\end{document}
